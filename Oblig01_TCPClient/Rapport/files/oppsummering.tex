\section{Oppsummering}
\subsection{Konklusjon}
Denne TCP klienten fungerer fint mot serveren til Christian og besvarer på oppgaven og dens protokoll. Denne oppgaven har vært ekstremt lærerikt, og det har vært en veldig gøy opplevelse å kode en slik klient, med tanke på at fagstoffet er nytt og ukjent. I tillegg så er det øving i rapportskriving og LaTeX, og det er jo veldig gøy. Det var kanskje litt vanskelig å forstå oppgaven riktig i begynnelsen, men det løste seg når Christian forklarte oppgaven grundig. Det var ikke vanskelig å kode klienten, selv om det har oppstått noen små problemer underveis. Heldigvis finnes det veldig mye bra dokumentasjon og forklaring på nett, så da løste det meste seg etter ett par søk på Google. Ellers er dette en perfekt oppgave og det har vært en god opplevelse å utvikle noe slikt.

\subsection{Referanser}
\renewcommand{\section}[2]{}
\begin{thebibliography}{50}
	\bibitem{christian}
		Christian Scott, Universitetet i Sørøst-Norge\\
		\textit{Avdelingsingeniør, lærer og guru ved USN Kongsberg}

	\bibitem{tcpbok}
		TCP/IP Sockets in C: Practical Guide for Programmers\\
		\textit{Elsevier Science, 2009. Skrevet av Michael Donahoo, Kenneth Calvert}

	\bibitem{knuthwebsite} 
		Java Tutorials: All About Sockets
		\\\textit{https://docs.oracle.com/javase/tutorial/networking/sockets/index.html}
\end{thebibliography}
	
\newpage
\subsection{Kildekode}
\begin{lstlisting}
	kode her
\end{lstlisting}
