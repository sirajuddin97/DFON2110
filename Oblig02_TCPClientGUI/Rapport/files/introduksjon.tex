\section{Introduksjon}
\subsection{Oppgavebeskrivelse}
Denne oppgaven går ut på å lage en TCP klient som skal kommunisere med en TCP server på en gitt IP addresse og port. Kommunikasjonen foregår ved bruk av en spesifikk protokoll med noen spesifikke krav som må oppfylles, slik at bruker av klienten får kommunisert riktig med serveren. Dersom man har koblet seg til serveren og oppfyller alle krav til protokollen, så skal man holde en aktiv forbindelse med serveren ved at serveren pinger klienten og klienten må svare ved å ponge tilbake. På denne måten kan vi kommunisere med en server inntil enten klienten eller serveren stopper koblingen.

\subsection{Valg av operativsystem og språk}
TCP klienten har blitt programmert og utviklet i C++ med Qt Framework, og platformen som har blitt brukt er Linux. Klienten er programmert som GUI, altså programmet kjøres som et visuelt program og er ikke kommando-basert. Den forrige obligatoriske oppgaven ble programmert i Java, så derfor har utvikleren valgt å programmere denne nye obligatoriske oppgaven i C++, og for å gjøre oppgaven ekstra utfordrende og interessant så er den utviklet med GUI. Det som er utfordrende med GUI utvikling er at man må ta hensyn til mye annet enn hvis man kun kjører programmet i et kommando-basert grensesnitt, som for eksempel kunne håndtere tråder på riktig måte og kunne programmere vinduet godt nok til at det ikke krasjer eller trenger å restarte. Og ikke minst huske å rydde opp etter seg i minnet, siden slike GUI programmer bruker en del minne og dermed må dette tømmes for å ikke føre til minnelekkasje.
