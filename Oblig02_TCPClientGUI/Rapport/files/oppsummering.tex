\section{Oppsummering}
\subsection{Konklusjon}
Denne oppgaven har vært ekstremt lærerikt, og det har vært en veldig gøy opplevelse å kode en slik klient, med tanke på at fagstoffet er nytt og ukjent. I tillegg så er det øving i rapportskriving og LaTeX, og det er jo veldig gøy. Det er selvfølgelig alltid morro å kode i C++ og Qt med tanke på at det er utrolig mye man kan få til. Det har vært litt problemer og misforståelser med leveringsfrister og justeringer underveis, og det syntes jeg skulle vært bedre planlagt på forhånd. Men jeg har ingenting å klage på - jeg har kost meg masse og lært en del nye ting, og det setter jeg pris på.

Jeg har brukt mye tid på koden med tanke på at den er laget med GUI, og derfor brukt mindre tid på rapporten. Jeg skulle kanskje starte mye tidligere på rapporten, for da hadde jeg rukket å skrive en ordentlig og fin rapport. Ellers er dette en perfekt oppgave og det har vært en god opplevelse å utvikle noe slikt.

\subsection{Referanser}
\renewcommand{\section}[2]{}
\begin{thebibliography}{50}
	\bibitem{christian}
		Christian Scott, Universitetet i Sørøst-Norge\\
		\textit{Avdelingsingeniør, lærer og guru ved USN Kongsberg}

	\bibitem{tcpbok}
		TCP/IP Sockets in C: Practical Guide for Programmers\\
		\textit{Elsevier Science, 2009. Skrevet av Michael Donahoo, Kenneth Calvert}
\end{thebibliography}

\newpage
\subsection{Kildekode}
\textbf{main.cpp:}
\begin{lstlisting}
	#include <QApplication>
	#include "client.h"

	int main(int argc, char *argv[]){
		QApplication a(argc, argv);
		TCPClient client;
		client.show();

		return a.exec();
	}
\end{lstlisting}

\textbf{client.h:}
\begin{lstlisting}
	#ifndef CLIENT_H
	#define CLIENT_H
	#include <QMainWindow>
	#include <QDateTime>
	#include "socket.h"

	namespace Ui{
		class TCPClient;
	}

	class TCPClient : public QMainWindow{
		Q_OBJECT

	public:
		explicit TCPClient(QWidget *parent = nullptr);
		~TCPClient();
		void addLog(QString);

	private slots:
		void on_connectButton_clicked();
		void on_clearButton_clicked();

	private:
		Ui::TCPClient *ui;
		Socket socket;
	};

	#endif
\end{lstlisting}

\newpage
\textbf{socket.h:}
\begin{lstlisting}
	#ifndef SOCKET_H
	#define SOCKET_H
	#include <iostream>
	#include <string>
	#include <string.h>
	#include <sys/socket.h>
	#include <arpa/inet.h>
	#include <unistd.h>

	namespace Server{
		struct serverInfo{
			int sock, port;
			std::string ip;
			bool isConnected;
			sockaddr_in serv_addr;
		};

		struct studentInfo{
			char* number = new char[6];
		};

		enum MessageID{
			REQUEST_PORT = 0x01,
			RECEIVE_PORT = 0x02,
			PING = 0x03,
			PONG = 0x04,
			QUIT = 0x05
		};

		enum Errors{
			SOCKET_ERROR,
			INVALID_CONNECTION,
			INVALID_STUDNR,
			INVALID_PORTREQUEST,
			INVALID_PORTRESPONSE,
			PING_ERROR,
			PONG_ERROR
		};
	};

	class Socket{
	public:    
		Socket() : server{ 0, 0, "NULL", false } {}
		void makeConnection(std::string, int);
		void abortConnection();
		bool getConnectionStatus() const{ return server.isConnected; }
		char* getStudentNumber() const{ return student.number; }
		void verifyStudent(std::string);
		void requestPort();
		short receivePort();
		Server::MessageID portResponse();
		void pongServer();

	private:
		Server::serverInfo server;
		Server::studentInfo student;
	};

	#endif
\end{lstlisting}

\textbf{client.cpp:}
\begin{lstlisting}
	#include "client.h"
	#include "ui_client.h"

	TCPClient::TCPClient(QWidget *parent) : QMainWindow(parent), ui(new Ui::TCPClient){
		ui->setupUi(this);
	}

	TCPClient::~TCPClient(){
		delete ui;
	}

	void TCPClient::addLog(QString text){
		QDateTime dateTime = dateTime.currentDateTime();
		QString dateTimeString = dateTime.toString("hh:mm:ss");
		ui->logBox->appendPlainText("[" + dateTimeString + "] " + text);
	}

	void TCPClient::on_connectButton_clicked(){
		if(!ui->ipBox->text().isEmpty() && !ui->portBox->text().isEmpty() && !ui->studBox->text().isEmpty()){
			std::string tempIP = ui->ipBox->text().toStdString();
			int tempPort = ui->portBox->text().toInt();
			std::string studNumber = ui->studBox->text().toStdString();

			if(!socket.getConnectionStatus()){
				try{
					socket.verifyStudent(studNumber);
					socket.makeConnection(tempIP, tempPort);
					addLog("Connected to " + ui->ipBox->text() + " on port " + ui->portBox->text());

					socket.requestPort();
					short newPort = socket.receivePort();
					addLog("Received a new port from server (" + QString::number(newPort) + ")");
					addLog("Disconnecting from current port (" + ui->portBox->text() + ")");
					socket.abortConnection();

					socket.makeConnection(tempIP, newPort);
					addLog("Re-connected to " + ui->ipBox->text() + " on port " + QString::number(newPort));
					ui->connectButton->setText("Disconnect");

					Server::MessageID response = socket.portResponse();
					while(response == Server::MessageID::PING){
						addLog("Server is pinging the client, returning a pong in response");
						socket.pongServer();
						response = socket.portResponse();
					}
					if(response == Server::MessageID::QUIT) addLog("Connection aborted! Disconnecting from the server");
					socket.abortConnection();
					ui->connectButton->setText("Connect");
				}
				catch(Server::Errors e){
					switch(e){
						case Server::INVALID_STUDNR: addLog("Invalid student number!"); break;
						case Server::SOCKET_ERROR: addLog("Failed to create a socket!"); break;
						case Server::INVALID_CONNECTION: addLog("Failed to connect to the server!"); break;
						case Server::INVALID_PORTREQUEST: addLog("Failed to receive a new port from the server!"); break;
						case Server::PING_ERROR: addLog("Server could not ping the client properly!"); break;
						case Server::PONG_ERROR: addLog("Client could not pong the server properly!"); break;
						case Server::INVALID_PORTRESPONSE: addLog("Invalid port response!"); break;
					}
				}
				catch(...){ addLog("Unknown error occured!"); }
			}
			else{
				socket.abortConnection();
				addLog("Disconnected from the server");
				ui->connectButton->setText("Connect");
			}
		}
		else addLog("Please enter the server IP address, port number and student number!");
	}

	void TCPClient::on_clearButton_clicked(){
		ui->logBox->clear();
	}
\end{lstlisting}

\newpage
\textbf{socket.cpp:}
\begin{lstlisting}
	#include "socket.h"

	void Socket::makeConnection(std::string ip, int port){
		if(ip == "lekeplass") server.ip = "158.36.70.56"; else server.ip = ip;
		server.port = port;
		server.sock = socket(AF_INET, SOCK_STREAM, 0);
		if(server.sock == 0) throw Server::SOCKET_ERROR;

		server.serv_addr.sin_addr.s_addr = inet_addr(server.ip.c_str());
		server.serv_addr.sin_family = AF_INET;
		server.serv_addr.sin_port = htons(server.port);
		int c = connect(server.sock, (sockaddr*)& server.serv_addr, sizeof(server.serv_addr));
		if(c < 0) throw Server::INVALID_CONNECTION;
		server.isConnected = true;
	}

	void Socket::abortConnection(){
		close(server.sock);
		server.sock = 0;
		server.port = 0;
		server.ip = "NULL";
		server.isConnected = false;
	}

	void Socket::verifyStudent(std::string studNumber){
		if(studNumber.length() != 6) throw Server::INVALID_STUDNR;
		student.number = (char*)studNumber.c_str();
	}

	void Socket::requestPort(){
		char buffer[10];
		char messageID = Server::MessageID::REQUEST_PORT;
		const short packageSize = sizeof(packageSize) + sizeof(messageID) + strlen(student.number);

		memset(buffer, 0, sizeof(buffer));
		memcpy(buffer, &packageSize, sizeof(packageSize));
		memcpy(buffer+2, &messageID, sizeof(messageID));
		memcpy(buffer+3, student.number, strlen(student.number));

		int s = send(server.sock, buffer, sizeof(buffer), 0);
		if(s < 0) throw Server::INVALID_PORTREQUEST;
	}

	short Socket::receivePort(){
		char buffer[10];
		memset(buffer, 0, sizeof(buffer));
		int r = recv(server.sock, buffer, sizeof(buffer), 0);
		if(r < 0) throw Server::INVALID_PORTREQUEST;

		short newPort;
		memcpy(&newPort, buffer+3, sizeof(newPort));
		return ntohs(newPort);
	}

	Server::MessageID Socket::portResponse(){
		char buffer[10];
		memset(buffer, 0, sizeof(buffer));
		int r = recv(server.sock, buffer, sizeof(buffer), 0);
		if(r < 0) throw Server::PING_ERROR;

		switch(buffer[2]){
			case Server::MessageID::PING:{
				return Server::MessageID::PING;
			}
			case Server::MessageID::PONG:{
				return Server::MessageID::PONG;
			}
			case Server::MessageID::QUIT:{
				return Server::MessageID::QUIT;
			}
			default:{
				throw Server::INVALID_PORTRESPONSE;
			}
		}
	}

	void Socket::pongServer(){
		char buffer[10];
		char messageID = Server::MessageID::PONG;
		const short packageSize = sizeof(packageSize) + sizeof(messageID);

		memset(buffer, 0, sizeof(buffer));
		memcpy(buffer, &packageSize, sizeof(packageSize));
		memcpy(buffer+2, &messageID, sizeof(messageID));

		int s = send(server.sock, buffer, sizeof(buffer), 0);
		if(s < 0) throw Server::PONG_ERROR;
	}
\end{lstlisting}
